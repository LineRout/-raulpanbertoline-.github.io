
\documentclass[10pt, a4paper]{article}

% --- PACOTES NECESSÁRIOS ---
\usepackage[utf8]{inputenc}        % Permite o uso de acentos (padrão no Overleaf)
\usepackage[brazil]{babel}         % Configura para o português do Brasil
\usepackage{geometry}              % Para ajustar as margens da página
\usepackage{hyperref}              % Para criar links clicáveis (email, LinkedIn)
\usepackage{titlesec}              % Para customizar os títulos das seções
\usepackage{enumitem}              % Para customizar listas (bullet points)

% --- CONFIGURAÇÕES DA PÁGINA E LAYOUT ---

% Define as margens
\geometry{
    a4paper,
    left=2cm,
    right=2cm,
    top=2cm,
    bottom=2cm
}

% Remove a numeração das páginas
\pagestyle{empty}

% Remove o recuo (indentação) do início dos parágrafos
\setlength{\parindent}{0pt}

% Configura o estilo das seções
\titlespacing*{\section}{0pt}{12pt}{6pt} % Espaçamento: {comando}{espaço à esq.}{espaço acima}{espaço abaixo}
\titleformat{\section}
    {\Large\bfseries\scshape\raggedright} % Formato: Letras maiúsculas pequenas, negrito, tamanho grande
    {}
    {0em}
    {}
    [\titlerule] % Adiciona uma linha horizontal abaixo do título da seção

% Configura o estilo das listas (bullet points) para serem mais compactas
\setlist[itemize]{
    leftmargin=*,
    label=\textbullet,
    itemsep=2pt,
    topsep=3pt
}

% Define uma cor para os links
\hypersetup{
    colorlinks=true,
    linkcolor=black,
    urlcolor=blue,
    pdftitle={Currículo - Raul Pan Bertoline},
    pdfauthor={Raul Pan Bertoline},
}


% --- INÍCIO DO DOCUMENTO ---
\begin{document}

% --- CABEÇALHO ---
\begin{center}
    {\Huge \scshape RAUL PAN BERTOLINE} % Nome em letras maiúsculas pequenas e tamanho grande
    \vspace{4pt} % Pequeno espaço vertical

    Cornélio Procópio - PR | Fernando Prestes - SP \\
    (16) 99702-2903 \textbullet{}
    \href{mailto:raulbertoline@gmail.com}{raulbertoline@gmail.com} \textbullet{}
    \href{https://www.linkedin.com/in/raul-pan-bertoline}{linkedin.com/in/raul-pan-bertoline}
\end{center}


% --- OBJETIVO ---
\section*{Objetivo}
Estudante de Engenharia de Computação proativo e dedicado, buscando uma oportunidade de estágio para aplicar e aprimorar meus conhecimentos em desenvolvimento de software. Tenho grande interesse em contribuir para projetos desafiadores e desenvolver soluções tecnológicas inovadoras, com preferência pela área bancária, mas aberto a novos aprendizados em diferentes setores da tecnologia.


% --- FORMAÇÃO ACADÊMICA ---
\section*{Formação Acadêmica}
\textbf{Bacharelado em Engenharia de Computação} \hfill \textbf{Previsão de Conclusão: Junho/2026} \\
\textit{Universidade Tecnológica Federal do Paraná (UTFPR) – Campus Cornélio Procópio}


% --- PROJETOS RELEVANTES ---
\section*{Projetos Relevantes}
\textbf{Quiz Interativo de Eletrônica (Projeto Social ELLP)}
\begin{itemize}
    \item Desenvolvimento de quiz interativo no estilo Kahoot, com foco em eletrônica, para ensino lúdico de crianças atendidas pelo projeto.
\end{itemize}

\vspace{5pt} % Espaço entre os projetos

\textbf{Funcionalidade de Coleta de Ideias (Meninas Digitais – UTFPR-CP)}
\begin{itemize}
    \item Implementação de funcionalidade para coleta e organização de ideias no site do projeto.
    \item Foco em usabilidade e design da interface para garantir uma experiência de usuário clara e intuitiva.
\end{itemize}


% --- HABILIDADES ---
\section*{Habilidades}
\begin{itemize}
    \item \textbf{Desenvolvimento:} Desenvolvimento Web Front-end, Desenvolvimento Mobile.
    \item \textbf{Design \& UX/UI:} Design de Interfaces, Prototipagem de Aplicações.
    \item \textbf{Marketing Digital:} Gestão de Tráfego Pago.
    \item \textbf{Competências Comportamentais:} Comunicação, Trabalho em Equipe, Resolução de Problemas, Proatividade, Adaptabilidade.
\end{itemize}


% --- ATIVIDADES EXTRACURRICULARES ---
\section*{Atividades Extracurriculares}
\textbf{Monitor no Projeto Social ELLP}
\begin{itemize}
    \item Atuação como monitor voluntário, auxiliando no ensino de conceitos de tecnologia e eletrônica para crianças.
    \item Desenvolvimento de materiais didáticos e atividades práticas, incluindo o quiz interativo.
\end{itemize}


% --- IDIOMAS ---
\section*{Idiomas}
\begin{itemize}
    \item \textbf{Inglês:} Avançado (Formado por Wizard Idiomas)
\end{itemize}


\end{document}
% --- FIM DO DOCUMENTO ---
